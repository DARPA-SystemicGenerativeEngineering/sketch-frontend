\section{Advanced Usage and Diagnostics}


\subsection{Temporary Files and Frontend Backend Communication}

The sketch frontend communicates with the solver through temporary files. 
By default, these files are named after the sketch you are solving and 
are placed in your temporary directory and deleted right afterwards. One 
unfortunate consequence of this is that if you run two instances of sketch at the same 
time on the same sketch (or on two sketch files with the same name), the temporary file
can get corrupted, leading to a compiler crash. In order to avoid this problem, you can use the flag 
\C{--fe-output} to direct the frontend to put the temporary files in a different directory. 

\flagdoc{fe-output}{Temporary output directory used to communicate with backend solver.}

Also, if you are doing advanced development on the system, you will sometimes want to keep 
the temporary files from being deleted. You can do this by using the \C{--fe-keep-tmp} flag.

\flagdoc{fe-keep-tmp}{Keep intermediate files used by the sketch frontend to communicate with the solver.}
