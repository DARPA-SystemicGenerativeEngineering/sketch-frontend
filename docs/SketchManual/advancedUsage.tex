\subsection{Custom Code Generators}

For many applications, the user's goal is not to generate C code, but instead to derive code details that will later be used by other applications. In order to simplify this process, \Sk{} makes it easy to create custom code generators that will be invoked by the sketch compiler at code generation time. 

Custom code generators must implement the \C{FEVisitor} interface defined in the \C{sketch.compiler.ast.core} package and must have a default constructor that the compiler can use to instantiate them. In order to ask the compiler to use a custom code generator, you must label your custom code generator with the \C{@CodeGenerator} annotation. You must then package your code generator together with any additional classes it uses into a single jar file, and you must tell \Sk{} to use this jar file by using the flag \C{--fe-custom-codegen}. 

\flagdoc{--fe-custom-codegen}{Flag takes as an argument the name of a jar file and forces \Sk{} to use the first code generator it finds in that file.}


To illustrate how to create a custom code generator, the \Sk{} distribution includes a folder called \C{sketch-frontend/customcodegen} that contains a custom code generator called \C{SCP} that simply pretty-prints the program to the terminal. In order to get \Sk{} to use this class as a code generator, follow these simple steps:

\begin{itemize}
\item From the \C{sketch-frontend} directory, compile the code generator by running \newline
 \C{> javac -cp sketch-}\version{}\C{-noarch.jar customcodegen/SCP.java}
\item Create a jar file by running \newline
\C{> jar -cvf customcodegen.jar customcodegen/}
\item Try out your new code generator by running \newline
\C{> sketch --fe-custom-codegen customcodegen.jar test/sk/seq/miniTest1.sk}
\end{itemize}

When you run, you should see the following messages in the output:
\begin{lstlisting}
Class customcodegen.SCP is a code generator.
Generating code with customcodegen.SCP
(followed by the pretty-printed version of your code).
\end{lstlisting}



\subsection{Temporary Files and Frontend Backend Communication}

The sketch frontend communicates with the solver through temporary files. 
By default, these files are named after the sketch you are solving and 
are placed in your temporary directory and deleted right afterwards. One 
unfortunate consequence of this is that if you run two instances of sketch at the same 
time on the same sketch (or on two sketch files with the same name), the temporary file
can get corrupted, leading to a compiler crash. In order to avoid this problem, you can use the flag 
\C{--fe-output} to direct the frontend to put the temporary files in a different directory. 

\flagdoc{fe-output}{Temporary output directory used to communicate with backend solver.}

Also, if you are doing advanced development on the system, you will sometimes want to keep 
the temporary files from being deleted. You can do this by using the \C{--fe-keep-tmp} flag.

\flagdoc{fe-keep-tmp}{Keep intermediate files used by the sketch frontend to communicate with the solver.}
